%% projeto_pesquisa.tex, baseado no modelo do projeto abnTeX2
%%
%% abtex2-modelo-projeto-pesquisa.tex, v-1 PFC 1 2016
%% Copyright 2012-2015 by abnTeX2 group at http://www.abntex.net.br/ 
%%
%% This work consists of the files abntex2-modelo-projeto-pesquisa.tex
%% and abntex2-modelo-references.bib
%%

% ------------------------------------------------------------------------
% ------------------------------------------------------------------------
% abnTeX2: Modelo de Projeto de pesquisa em conformidade com ABNT NBR 
% 15287:2011 Informação e documentação - Projeto de pesquisa -
% Apresentação 
% ------------------------------------------------------------------------ 
% ------------------------------------------------------------------------

\documentclass[
	% -- opções da classe memoir --
	12pt,				% tamanho da fonte
	openright,			% capítulos começam em pág ímpar (insere página vazia caso preciso)
	oneside,
    %twoside,			% para impressão em verso e anverso. Oposto a oneside
	a4paper,			% tamanho do papel. 
	% -- opções da classe abntex2 --
	%chapter=TITLE,		% títulos de capítulos convertidos em letras maiúsculas
	%section=TITLE,		% títulos de seções convertidos em letras maiúsculas
	%subsection=TITLE,	% títulos de subseções convertidos em letras maiúsculas
	%subsubsection=TITLE,% títulos de subsubseções convertidos em letras maiúsculas
	% -- opções do pacote babel --
	english,			% idioma adicional para hifenização
	french,				% idioma adicional para hifenização
	spanish,			% idioma adicional para hifenização
	brazil,				% o último idioma é o principal do documento
	]{abntex2}

% ---
% PACOTES
% ---

% ---
% Pacotes fundamentais 
% ---
\usepackage{lmodern}			% Usa a fonte Latin Modern
\usepackage[T1]{fontenc}		% Selecao de codigos de fonte.
\usepackage[utf8]{inputenc}		% Codificacao do documento (conversão automática dos acentos)
\usepackage{indentfirst}		% Indenta o primeiro parágrafo de cada seção.
\usepackage{color}				% Controle das cores
\usepackage{graphicx}			% Inclusão de gráficos
\usepackage{microtype} 			% para melhorias de justificação
\usepackage{pgfgantt}			% gráficos Gantt
% ---

% ---
% Pacotes adicionais, usados apenas no âmbito do Modelo Canônico do abnteX2
% ---
\usepackage{lipsum}				% para geração de dummy text
% ---

% ---
% Pacotes de citações
% ---
\usepackage[brazilian,hyperpageref]{backref}	 % Paginas com as citações na bibl
\usepackage[alf]{abntex2cite}	% Citações padrão ABNT

% --- 
% CONFIGURAÇÕES DE PACOTES
% --- 

% ---
% Configurações do pacote backref
% Usado sem a opção hyperpageref de backref
\renewcommand{\backrefpagesname}{Citado na(s) página(s):~}
% Texto padrão antes do número das páginas
\renewcommand{\backref}{}
% Define os textos da citação
\renewcommand*{\backrefalt}[4]{
	\ifcase #1 %
		Nenhuma citação no texto.%
	\or
		Citado na página #2.%
	\else
		Citado #1 vezes nas páginas #2.%
	\fi}%
% ---

% ---
% Informações de dados para CAPA e FOLHA DE ROSTO
% ---
\titulo{Implantação de ligação ferroviária ao longo do curso do rio Tietê na Região Metropolitana de São Paulo: comparação com a infraestrutura existente ao longo do curso do rio Pinheiros}
\autor{Caio César Carvalho Ortega}
\local{São Bernardo do Campo-SP}
\data{2018}
\tipotrabalho{Projeto para escrita de artigo (Graduação)}
% O preambulo deve conter o tipo do trabalho, o objetivo, 
% o nome da instituição e a área de concentração 
\preambulo{Projeto para escrita de artigo apresentado ao curso de Bacharelado em Ciências e Humanidades, como requisito para obtenção do grau final na disciplina de Práticas em Ciências e Humanidades}

\orientador[Orientadora:]{Profa. Dra. Paula Priscila Braga}


\renewcommand{\orientador}{Orientadora:}

\instituicao{Universidade Federal do ABC}


% ---

% ---
% Configurações de aparência do PDF final

% alterando o aspecto da cor azul
\definecolor{blue}{RGB}{41,5,195}

% informações do PDF
\makeatletter
\hypersetup{
     	%pagebackref=true,
		pdftitle={\@title}, 
		pdfauthor={\@author},
    	pdfsubject={\imprimirpreambulo},
	    pdfcreator={LaTeX with abnTeX2},
		pdfkeywords={abnt}{latex}{abntex}{abntex2}{projeto de pesquisa}, 
		colorlinks=true,       		% false: boxed links; true: colored links
    	linkcolor=blue,          	% color of internal links
    	citecolor=blue,        		% color of links to bibliography
    	filecolor=magenta,      		% color of file links
		urlcolor=blue,
		bookmarksdepth=4
}
\makeatother
% --- 

% --- 
% Espaçamentos entre linhas e parágrafos 
% --- 

% O tamanho do parágrafo é dado por:
\setlength{\parindent}{1.3cm}

% Controle do espaçamento entre um parágrafo e outro:
\setlength{\parskip}{0.2cm}  % tente também \onelineskip

% ---
% compila o indice
% ---
\makeindex
% ---

% ----
% Início do documento
% ----
\begin{document}

% Seleciona o idioma do documento (conforme pacotes do babel)
%\selectlanguage{english}
\selectlanguage{brazil}

% Retira espaço extra obsoleto entre as frases.
\frenchspacing 

% ----------------------------------------------------------
% ELEMENTOS PRÉ-TEXTUAIS
% ----------------------------------------------------------
% \pretextual

% ---
% Capa
% ---
\imprimircapa
% ---

% ---
% Folha de rosto
% ---
\imprimirfolhaderosto
% ---

% ---
% NOTA DA ABNT NBR 15287:2011, p. 4:
%  ``Se exigido pela entidade, apresentar os dados curriculares do autor em
%     folha ou página distinta após a folha de rosto.''
% ---

% ---
% inserir lista de ilustrações
% ---
\pdfbookmark[0]{\listfigurename}{lof}
\listoffigures*
\cleardoublepage
% ---

% ---
% inserir lista de tabelas
% ---
\pdfbookmark[0]{\listtablename}{lot}
\listoftables*
\cleardoublepage
% ---

% ---
% inserir lista de abreviaturas e siglas
% ---
\begin{siglas}
  \item[CPTM] Companhia Paulista de Trens Metropolitanos
  \item[RMSP] Região Metropolitana de São Paulo
\end{siglas}
% ---

% ---
% inserir lista de símbolos
% ---
%\begin{simbolos}
%  \item[$ \Gamma $] Letra grega Gama
%  \item[$ \Lambda $] Lambda
%  \item[$ \zeta $] Letra grega minúscula zeta
%  \item[$ \in $] Pertence
%\end{simbolos}
% ---

% ---
% inserir o sumario
% ---
\pdfbookmark[0]{\contentsname}{toc}
\tableofcontents*
\cleardoublepage
% ---


% ----------------------------------------------------------
% ELEMENTOS TEXTUAIS
% ----------------------------------------------------------
\textual

% ----------------------------------------------------------
% Introdução
% ----------------------------------------------------------
\chapter*[Introdução]{Introdução}
\addcontentsline{toc}{chapter}{Introdução}

Este projeto de pesquisa parte da seguinte pergunta: ``considerando que há uma ligação ferroviária que acompanha o curso do rio Pinheiros, faria sentido implantar outra acompanhando o curso do rio Tietê em, pelo menos, parte da Região Metropolitana de São Paulo?'', alimentando a hipótese de que, sim, seria possível, considerando as viagens realizadas pela Marginal Tietê, a possibilidade de aproximar a população do rio e a configuração formada pela Marginal Tietê paralela (ainda que em parte) às linhas 3-Vermelha, 11-Coral, 12-Safira, 7-Rubi e 8-Diamante. São as variáveis da hipótese: densidade de estações por km$^{2}$, densidade de conexões entre linhas que se cruzam por km$^{2}$, ganho de capacidade/hora (comparação entre vias, uma via de tráfego misto transporta 10 mil passageiros/hora, cada via férrea pode transportar mais 60 mil passageiros/hora).

% ----------------------------------------------------------
% Elementos Textuais
% ----------------------------------------------------------

%===== Seção de Apresentação =============

\chapter{Objetivos}

Os estudos aos quais tive acesso contribuem para compreender o território ligado ao rio Tietê e diferenciar a retificação, responsável por modificar o curso de ambos os rios e facilitar mudanças no tecido das vargens, em relação àquela realizada no Pinheiros, ligada aos negócios da Cia. City. Estes estudos, entretanto, não respondem minha pergunta. Não encontrei teses ou artigos que se debrucem especificamente sobre a possibilidade de construir uma ferrovia metropolitana ao longo do Tietê. Há de se considerar que o contexto que levou à construção do antigo ramal de Jurubatuba da finada Estrada de Ferro Sorocabana era distinto e São Paulo apresentava outra configuração populacional, além de pouco dinamismo ao longo do Pinheiros. A principal motivação da Sorocabana foi concorrer com a Santos-Jundiaí e oferecer outra ligação para acesso à Baixada Santista. Motivação esta que, hoje, nem mesmo dialoga com o atual uso da Linha 9-Esmeralda do Trem Metropolitano da CPTM.

Considerando as variáveis por mim colocadas, minha aposta inicial é que o artigo faria um levantamento histórico e em seguida partiria para uma breve análise do território e sua infraestrutura. Considerando o histórico e a infraestrutura. Minha aposta é que, sim, seria possível fazer, mas existem diferenças políticas e de atores consideráveis, para tanto, faz-se necessário identificar e registrar as principais diferenças na retificação dos dois rios; definir o volume de tráfego e de passageiros/hora ao longo das vias marginais, de forma a permitir comparações e; identificar o ganho de capacidade possível caso existisse uma linha metroferroviária similar à 9-Esmeralda acompanhando o rio Tietê, incluindo um mapeamento sobre a densidade de conexões e estações por km$^{2}$.

%===== Seção de Apresentação =============

\chapter{Justificativa}

\lipsum


%===== Seção de Metodologia ==============

\chapter{Metodologia}

\lipsum

%===== Seção de Metodologia ==============

\chapter{Cronograma}

\index{elementos textuais}
O cronograma tem por objetivo prever as ações distribuídas de acordo com o tempo previsto de pesquisa. O cronograma deve estar alinhado com os objetivos específicos e com a metodologia. Nos objetivos específicos tem-se “o que vou fazer”, na metodologia, “como vou fazer” e no cronograma, “quando vou fazer”.

A  \autoref{tabela_cronog} apresenta o cronograma de execução da pesquisa. 

\begin{table}[!htb]
\centering
\caption{Cronograma de Atividades}
\label{tabela_cronog}
\begin{tabular}{@{}llll@{}}
\toprule
\textit{Atividades}                  & \textit{Mês}       & \textit{Ano}   \\ \midrule
Revisão Sistemática                  & Julho              & 2016      \\
Análise de Trabalhos Relacionados    & Agosto             & 2016      \\
Construção da Arquitetura            & Setembro           & 2016      \\
Implementação do Sistema             & Outubro            & 2016      \\
Avalição e Testes                    & Fevereiro          & 2017      \\
                                     &                    &           \\ \bottomrule
\end{tabular}
\end{table}




% ----------------------------------------------------------
% Capitulo com exemplos de comandos inseridos de arquivo externo 
% ----------------------------------------------------------

%\include{abntex2-modelo-include-comandos}

% ---
% Finaliza a parte no bookmark do PDF
% para que se inicie o bookmark na raiz
% e adiciona espaço de parte no Sumário
% ---
\phantompart



% ----------------------------------------------------------
% ELEMENTOS PÓS-TEXTUAIS
% ----------------------------------------------------------
\postextual

% ----------------------------------------------------------
% Referências bibliográficas
% ----------------------------------------------------------
\bibliography{abntex2-modelo-references}

% ----------------------------------------------------------
% Glossário
% ----------------------------------------------------------
%
% Consulte o manual da classe abntex2 para orientações sobre o glossário.
%
%\glossary

% ----------------------------------------------------------
% Apêndices
% ----------------------------------------------------------

% ---
% Inicia os apêndices
% ---
%\begin{apendicesenv}
%
%% Imprime uma página indicando o início dos apêndices
%\partapendices
%
%% ----------------------------------------------------------
%\chapter{Exemplo de um apêndice A}
%% ----------------------------------------------------------
%
%
%
%% ----------------------------------------------------------
%\chapter{Exemplo de um apêndice B}
%% ----------------------------------------------------------
%
%
%\end{apendicesenv}
%% ---


% ----------------------------------------------------------
% Anexos
% ----------------------------------------------------------

% ---
% Inicia os anexos
% ---
%\begin{anexosenv}
%
%% Imprime uma página indicando o início dos anexos
%\partanexos
%
%% ---
%\chapter{Exemplo de um primeiro anexo}
%% ---
%
%
%% ---
%\chapter{Exemplo de um segundo anexo}
%% ---
%
%
%
%% ---
%\chapter{Exemplo de um terceiro anexo}
%% ---
%
%
%
%\end{anexosenv}

%---------------------------------------------------------------------
% INDICE REMISSIVO
%---------------------------------------------------------------------

\phantompart

\printindex


\end{document}
